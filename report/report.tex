\documentclass{article}
\usepackage{color}
\usepackage{graphicx}
\usepackage{float}
\usepackage{ctex}
\usepackage{tabularx}
\usepackage{geometry}
\usepackage{listings}
\usepackage{fontspec}
\usepackage{indentfirst}
\setlength{\parindent}{2em}
\geometry{a4paper,left=3cm,right=3cm,top=3.5cm,bottom=3.5cm}
\lstset{
    basicstyle={\fontspec{Consolas} \small},
    breaklines=true,
    keywordstyle={\color{red} \fontspec{Consolas Bold}},
    commentstyle={\color[RGB]{30,120,30} \fontspec{Consolas Italic}},
    tabsize=4
}

\begin{document}

% cover begin

\thispagestyle{empty}
\setcounter{page}{-1}

\begin{center}
    \includegraphics[width=0.5\paperwidth]{logo.png}
\end{center}

\vskip 20pt

\begin{center}
    \zihao{3}
    \textbf{本科实验报告}
\end{center}


\vskip 120pt

    \begin{center}
        \bfseries \zihao{-2}

        \begin{tabularx}{.7\textwidth}{>{\fangsong}l >{\fangsong}X<{\centering}}
            课程名称 & \uline{\hfill 编译原理 \hfill} \\
            姓名 &  \uline{\hfill \hfill} \\
            学院 &  \uline{\hfill \hfill} \\
            系 &  \uline{\hfill \hfill} \\
            专业 &  \uline{\hfill \hfill} \\
            学号 &  \uline{\hfill \hfill} \\
            指导教师 &  \uline{\hfill \hfill} \\
            ~ & ~\\
        \end{tabularx}
    \end{center}

\vskip 100pt

\begin{center}
    \zihao{3} \fangsong
    \today
\end{center}

% cover end


\newpage

\thispagestyle{empty}

\zihao{4}
\tableofcontents

\newpage

\zihao{-4}


\section*{序言}
\addcontentsline{toc}{section}{序言}

\subsection*{概述}
\addcontentsline{toc}{subsection}{概述}
\par paragraph 1 paragraph 1 paragraph 1 paragraph 1 paragraph 1 paragraph 1 paragraph 1 paragraph 1 paragraph 1 paragraph 1 paragraph 1 
\par paragraph 2

\subsection*{编译环境}
\addcontentsline{toc}{subsection}{编译环境}
\begin{itemize}
    \item Prebuilt environment: Windows Sub Linux Ubuntu 20.04
    \item Lexer: flex 2.6.4
    \item Parser: bison 3.5.1
    \item Framework: LLVM 12.0.0
\end{itemize}

\subsection*{文件结构}
\addcontentsline{toc}{subsection}{文件结构}
\begin{lstlisting}
Root
│  build.sh  //编译compiler
│  clean.sh  //清理产生的compiler及中间文件
│  run.sh  //运行compiler并将产生的LLVM IR编译为可执行文件
│
├─bin
│  └─ main  //预编译的compiler
├─src
│  │  main.cc  //入口
│  │  Parser.cc  //bison产生的中间文件
│  │  Parser.hh  //bison产生的中间文件
│  │  Scanner.cc  //flex产生的中间文件
│  │
│  ├─ast
│  │      ast.hh  //include-all header
│  │      ast_base.cc
│  │      ast_base.hh //基类node
│  │      ast_const.cc
│  │      ast_const.hh  //常量node,包含各种常量从token到llvm::Value的转换
│  │      ast_expr.cc
│  │      ast_expr.hh  //表达式node,包含表达式中会出现的所有token的解析
│  │      ast_function.cc
│  │      ast_function.hh  //函数node,包含函数的声明与定义
│  │      ast_routine.cc
│  │      ast_routine.hh  //例程node,包含routine和subroutine的实现
│  │      ast_stmt.cc
│  │      ast_stmt.hh  //语句node,包含赋值、流程控制、过程调用等
│  │      ast_type.cc
│  │      ast_type.hh  //类型node,包含类型的解析和从token到llvm::Type的转换
│  │      ast_variable.cc
│  │      ast_variable.hh  //变量node,包含变量的定义与内存分配
│  │
│  ├─irgen
│  │      generator.cc
│  │      generator.hh  //AST的操作者,实现AST的生成、AST的输出、IR的生成
│  │      table.cc
│  │      table.hh  //符号表的定义
│  │
│  ├─lexer
│  │      Scanner.l  //词法分析器
│  │
│  ├─logger
│  │      logger.cc
│  │      logger.hh  //输出log的工具
│  │
│  └─parser
│         Parser.yy  //语法分析器
│
└─test
        test1.spl  //基本语句的测试
        test2.spl  //给定test
        test2ans.cc  //给定test的c++转写,用于测试结果
        test3.spl  //IO测试
        test4.spl  //给定test
        test4ans.cc
        test5.spl  //数组与记录的使用,函数引用传递的测试
        test6.spl  //给定test
        test6ans.cc

\end{lstlisting}

\subsection*{分工说明}
\addcontentsline{toc}{subsection}{分工说明}

\subsection*{总览}
\addcontentsline{toc}{subsection}{总览}

\begin{table}[H]
    \begin{tabular}{|l|l|l|}
        \hline
        \textbf{Compiler Component}    & \textbf{Status} & \textbf{Details}   \\ \hline
        Scanning                       & Flex            &                    \\ \hline
        Parsing                        & Bison           &                    \\ \hline
        Semantic Analysis              & AST             &                    \\ \hline
        Intermediate Code Generation   & LLVM IR         & LLVM IRBuilder     \\ \hline
        Intermediate Code Optimization & LLVM C++ API    & LLVM Pass Manager  \\ \hline
        Target Code Generation         & LLVM back-end   & Clang++            \\ \hline
        Target Code Optimization       & LLVM back-end   & Clang++            \\ \hline
        Other Optimization             & implemented     &                    \\ \hline
        Runtime Environment            & implemented     & 全局栈环境          \\ \hline
        Symbol Table                   & implemented     &                    \\ \hline
    \end{tabular}
\end{table}

\section{词法分析}
\subsection{正则表达式}
\subsection{实现原理和方法}

\section{语法分析}
\subsection{上下文无关语法}
\subsection{实现原理和方法}

\section{语义分析}
\subsection{实现方法}

\section{优化考虑}
\subsection{xxx阶段优化}

\section{代码生成}
\subsection{assign}
\begin{lstlisting}[language=c++,showstringspaces=false]
    int main()
    {
        // comment
        return 0;
    }
\end{lstlisting}

\section{测试案例}
\subsection{单句测试}
\subsection{组合测试}




\end{document}